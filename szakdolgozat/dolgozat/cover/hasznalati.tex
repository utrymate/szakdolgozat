\pagestyle{empty}

\noindent \textbf{\Large A melléklet tartalma}

\vskip 1cm

\noindent A dolgozat melléklete a következőket tartalmazza:
\begin{itemize}
\item \texttt{dolgozat.pdf:} A szakdolgozat PDF formátumban.
\item \texttt{dolgozat:} A szakdolgozat \LaTeX{} forráskódját tartalmazó jegyzék.
\item \texttt{program:} Az elkészített programok forráskódja.
\end{itemize}
A dolgozat az elkészített program forráskódjával együtt \texttt{GitHub}-on is megtalálható az alábbi címen:

\vskip 0.5cm

\texttt{https://github.com/utrymate/szakdolgozat}

\vskip 0.5cm

\noindent \textbf{\large A program futtatása}

\vskip 0.3cm

\noindent Ha telepítettük a 4-es fejezetben leírt alkalmazásokat, akkor a következőképpen tudjuk elindítani a programot:

\begin{itemize}
\item A \texttt{program} jegyzékben található \texttt{server} jegyzékbe lépünk a Parancssor segítségével.
\item Ha a \texttt{waitress}-t telepítettük, akkor az alábbi parancs indítja el a szervert: \texttt{waitress-serve --port=8000 server:api}
\item A klienset a \url{http://localhost:8000/graphEditor.html} oldalon keresztül érhetjük el.
\item Az adatbázis a \texttt{server} jegyzékben jön létre (ha még nem létezik) \texttt{mydb.sqlite} néven.
\end{itemize}
