\Chapter{Összefoglalás}

A dolgozat az üzleti folyamatok modellezésének megvalósítását célozta meg elméleti és gyakorlati példák bemutatásával, valamint a hozzá elkészített alkalmazással. Igyekezett felhasználóbarát központúvá tenni az alkalmazást annak működését és kinézetét illetően.

A program kliens oldalon JavaScriptet használt, amihez CSS-t alkalmazott, szerver oldalon pedig egy Python alapú mikrokeretrendszert, a Falcon-t. A gráf szerkesztés során használt adatokat SQLite adatbázisban tárolta.

Az alkalmazás a használat közbeni integritási feltételek ellenőrzésével próbált kiemelkedni a többi hasonló gráf szerkesztő alkalmazás közül. Azonban még sok továbbfejlesztési lehetőség tárul elénk, hiszen ezen ellenőrzések sokaságának köszönhetően további, bonyolultabb validálási feltételeket is meghatározhatunk a jövőben.

Ha lesz legalább egy szoftverfejlesztő csapat, aki felkapja az alkalmazást, és továbbfejleszti az integritási feltételeket ellenőrző részét, akkor már megérte elkészíteni az ehhez alapul szolgáló gráf szerkesztő programot.