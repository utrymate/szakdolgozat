\Chapter{JavaScript implementáció}

\Section{A kliens program működése}

A kliens program a következőképpen működik: A graphEditor.html dokumentum betöltése után létrehozzuk az Editor osztály egy példányát. Ezután inicializáljuk a szerkesztőt egy függvény segítségével. Itt beállítjuk a program működéséhez elengedhetetlen adattagokat, események kezelését. lyen adattagok a graph, canvas, context, themes.

Végül egy metódus segítségével átméretezzük az adott böngészőablak méretéhez arányosan biztosítva így, hogy bármilyen méretű képernyőn használható legyen a program.

%todo

Az indítás után a program egy aszinkron GET kéréssel lekéri az adatokat a szervertől JSON formátumban. Az adatokból létrehozza a megfelelő osztályokat a konstruktor paraméterekkel. Ezután a csomópontokat hozzáadja a nodes adattaghoz. Ha ez sikerült, a függvény egy Promise-szal (ígéret) jelzi, és lekérhetjük az összekötő vonalakat. Mivel aszinkron az előző kérés is, előfordulhat olyan, hogy az összekötő vonalak hamarabb megérkeznek mint maguk a csomópontok, és ez a programban végzetes hibához vezetne. Ezért van szükség a Promise-ra. 

\begin{javascript}
function GetEditorData() {
   fetch(apiURL+'/nodes').then(response => response.json())
       .then(data => {
           console.log(data);
           data.map(node => {
               let newNode = {};
               switch (node["type"]) {
                   case "rectangle":
                       newNode = new Rectangle(node["x"],node["y"],node["id"],node["height"], node["width"], node["text"], editor.graph.themes[editor.graph.selectedtheme].rectangleColor);
                       break;
                   case "circle":
                       newNode = new Circle(node["x"],node["y"],node["id"],node["radius"], node["text"], editor.graph.themes[editor.graph.selectedtheme].circleColor);
                       break;
                   case "diamond":
                       newNode = new Diamond(node["x"],node["y"],node["id"],node["height"], node["width"], node["text"], editor.graph.themes[editor.graph.selectedtheme].diamondColor);
                       break;
                   case "start":
                       newNode = new Start(node["x"],node["y"],node["id"],node["radius"], node["text"]);
                       break;
                   case "end":
                       newNode = new End(node["x"],node["y"],node["id"],node["radius"], node["text"]);
                       break;
               }
               editor.graph.nodes.push(newNode)
           })
       }).then(r => {
           //itt mar betoltottuk a nodes-okat

//*

       fetch(apiURL+'/edges').then(response =>response.json())
           .then(data => {
               data.map(elem =>{
                   let from = editor.getNode(elem.edgefrom);
                   let to = editor.getNode(elem.edgeto);
                   let newEdge = new Edge(from, to,elem.id);
                   editor.graph.edges.push(newEdge);
               })
           })
})

\end{javascript}
A //*-gal jelölt rész után egy másik GET kéréssel lekérjük az összekötő vonalakat, és a megfelelő adatokból létrehozzuk az Edge osztály példányait és hozzáadjuk őket az edges adattaghoz, így a következő rajzolásnál már látszódnak a gráfok. 

A mentés gombra kattintva először átalakítjuk az nodes és edges adattagokban található osztályokat JSON adattá, majd egy aszinkron POST kéréssel mentjük az adatokat a szerveren.
Az csomópontokat megvalósító osztályokat univerzálisan kezeljük, amilyen tulajdonsággal nem rendelkezik egyszerűen 0-ra állítjuk, úgyse foglalkozunk vele.

\begin{javascript}
function saveEditorData() {
   let example = {"type":"circle","x":100,"y":200,"id":1,"text":"valami", "height":0,"width":0, "radius":50};
   let nodes = [];
   let edges = [];
   let type="";
   editor.graph.nodes.map(node => {
           switch (node.constructor.name ){
           case "Rectangle":
               type = "rectangle";
               break;
           case "Circle":
               type = "circle";
               break;
           case "Diamond":
               type = "diamond";
               break;
           case "Start":
               type = "start";
               break;
           case "End":
               type = "start";
               break;
       }

       let nodeJSON = {
           "type": type,
           "x": node.x,
           "y": node.y,
           "id": node.id,
           "text":node.text,
           "height": node.height ? node.height : 0,
           "width": node.width ? node.width : 0,
           "radius": node.radius ? node.radius : 0,
           //"theme": editor.graph.selectedtheme
       };
       nodes.push(nodeJSON)

   });
   editor.graph.edges.map(edge =>{
       let edgeJSON = {
           "id": edge.id,
           "from": edge.from.id,
           "to": edge.to.id
       };
       edges.push(edgeJSON);
   });
   console.log(JSON.stringify(nodes));
   fetch(apiURL+'/nodes', {
       method: 'POST',
       body: JSON.stringify(nodes)
   }).then(response => response.json()).then(data => {
       //ha valami ertelmeset is kapunk itt elerjuk
   }).then(r =>{
       fetch(apiURL+'/edges', {
           method: 'POST',
           body: JSON.stringify(edges)
       }).then(response => response.json()).then(data => {
           //ha valami ertelmeset is kapunk itt elerjuk
       });
   })
\end{javascript}

 
Az eseménykezelés a következőképp zajlik: A billentyűk lenyomása és felengedéséhez az editor megfelelő metódusát rendeljük.

\begin{javascript}
window.addEventListener("keydown", editor.keyDown, false);
window.addEventListener("keyup", editor.keyUp, false);
editor.graph.canvas.addEventListener("mousedown", editor.mouseDown.bind(editor), false);
editor.graph.canvas.addEventListener("mousemove", editor.mouseMove.bind(editor), true);
editor.graph.canvas.addEventListener("mouseup", editor.mouseUp.bind(editor), false);
editor.graph.canvas.addEventListener("mousewheel", editor.mouseWheel.bind(editor), false);
\end{javascript}
 
Például a mouseDown eseménykezelő metódus úgy működik hogy ha lenyomjuk az egér bal gombját, a függvény újra rajzoltatja a gráfokat, majd kiszámítja hogy az egér rajta van-e valamely csomóponton, ha rajta van az lesz az éppen kijelölt csomópont, és a csomópont osztályban jelez, hogy húzható (drag) állapotba került, azaz az éppen kijelölt csomópont követi az egérmutató mozgását.

\begin{javascript}
mouseDown(event)
{
   this.graph.clear();
   this.graph.draw();
   const mouse = this.calcMouseEvent(event);
   this.graph.selectedIndex = null;
   for (let i = 0; i < this.graph.nodes.length; ++i) {
       let node = this.graph.nodes[i];
       if (node.isMouseOnNode(mouse['x'], mouse['y'])){
           this.graph.selectedIndex = i;
       }
   }
   this.graph.dragStart = {
       x: event.clientX - this.graph.canvasPosition.left,
       y: event.clientY - this.graph.canvasPosition.top
   };
   this.graph.drag = true;
}
\end{javascript}

 
Ha a megfelelő helyre húztuk a csomópontot és felengedtük az egér bal gombját, a mouseUp eseménykezelő metódus jelzi a csomópontokat kezelő osztálynak, hogy a csomópont már nem húzható, azaz nem követi tovább az egeret.

\begin{javascript}
mouseUp(event, context)
{
   this.graph.clear();
   this.graph.draw();
   const mouse = this.calcMouseEvent(event);
   this.graph.drag = false;
}
\end{javascript}

Az csomópontokat úgy tudjuk összekötni, hogy a kiinduló (forrás) csomópontra rákattintunk miközben nyomva tartjuk a \texttt{shift} billentyűt. Egy másik (a cél)csomópontra kattintva, miközben nyomva tartjuk az \texttt{alt} billentyűt, kijelölhetjük a összekötendő csomópontot. A \texttt{ctrl} billentyűvel törölni tudjuk a csomópontot.

\begin{javascript}
if (event.shiftKey === true) {
   this.graph.drawEdge['from'] = this.graph.nodes[selectedIndex];
}
if (event.altKey === true) {
   this.graph.drawEdge["to"] = this.graph.nodes[selectedIndex];//node["id"];
   this.addEdge(this.graph.drawEdge["from"], this.graph.drawEdge["to"]);
   this.graph.clear();
   this.graph.draw();
}
if (event.ctrlKey === true) {
   for (var j = 0; j < this.graph.edges.length; ++j) {
       if(this.graph.nodes[selectedIndex] === this.graph.edges[j].to || this.graph.nodes[selectedIndex] === this.graph.edges[j].from)
       {
           this.graph.edges.splice(j, 1);
           j = j-1;
       }
   }
   this.graph.nodes.splice(selectedIndex, 1);
   this.graph.clear();
   this.graph.draw();
}
\end{javascript}

Az automatikus és manuális méretnövelés a következőképpen működik: 
Minden csomópontnak van megfelelő metódusa a feladathoz, ami a csomópont saját tulajdonsága alapján elvégzi a feladatot. Például a \textit{widen(length)} metódus elfogad egy paramétert, amely ha túl nagynak bizonyul, növeli a csomópont méretét. A téglalap esetében a szélességet növeljük:

\begin{javascript}
widen(length){
   if(15 > (this.width/length))
   {
       this.width = parseInt(this.width) + 14;
   }
}
\end{javascript}

Viszont például egy körnél a sugarat növeljük:

\begin{javascript}
widen(length){
   if(8 > this.radius / length)
   {
       this.radius += 7;
   }
}
\end{javascript}

Átméretezni egy csomópontot a resize metódussal lehet, itt is hasonlóan a csomópont tulajdonságaihoz megfelelő logikával:

\begin{javascript}
resize(){
   this["width"] = parseInt(document.getElementById("width").value);
   this["height"] =  parseInt(document.getElementById("height").value);
}
\end{javascript}
\Section{Felhasználói kézikönyv a programhoz}

Ez az alfejezet írja le a már korábban többször is említett konkrét kezelését az elkészült gráfszerkesztőnek.

\SubSection{Csomópont hozzáadása}

A bal oldali eszköztárban rákattintunk a hozzáadandó elemre, így az megjelenik a szerkesztő felületen egy előre definiált helyen. Ezt követően az adott csomóponton a bal egérgombot nyomva tartva húzással tudjuk azt mozgatni azt. Fontos, hogy csak a szerkesztő felület területén belül, annak szélét nem érheti el, tehát nem tudjuk "levágni" a csomópont egy részét a canvas-on túli területre való mozgatással.

\SubSection{Csomópontok összekötése vonallal}

Az összekötésnél figyelembe kell vennünk, hogy a program megkülönböztet forrás- és cél csomópontot, hiszen ezalapján dönti el, hogy a vonal melyik végén lesz a nyíl (vagyis melyik csomópontból melyik másik következik).

\textbf{Forrás csomópont kijelölése:} A szerkesztő felületen a \texttt{Shift} gomb lenyomása mellett az adott csomópontra való kattintás fogja kijelöni a forrás csomópontot.

\textbf{Célcsomópont kijelölése:} A Shift gombot elengedve, helyette az \texttt{Alt} gomb lenyomása mellett egy másik csomópontra való kattintás az egérrel fogja meghatározni a célcsomópontot, vagyis azt, hogy hová lesz bekötve az adott vonal.

Ezt követően már meg is jeleníti a program a kívánt összeköttetést a két csomópont között.

\SubSection{Csomópont törlése}

A törlést a \texttt{Ctrl} gomb nyomva tartása közben a törölni kívánt csomópontra való kattintás teszi lehetővé. Ilyenkor a csomóponthoz kapcsolt vonalak is törlésre kerülnek a csomóponttal együtt.

\SubSection{Szöveg hozzáadása csomóponthoz}

Amelyik csomópontra szöveget szeretnénk írni, azon egy dupla kattintás meg fogja nyitni az eszköztár aljában a szöveg hozzáadásához szükséges részt. Ez a rész addig nem látható, amíg rá nem kattintunk egy csomópontra.

Az itt beírt szöveg azonnal megjelenik az adott csomóponton, illetve ha a beírt szöveg hossza meghaladja a csomópont szélességét, akkor az automatikusan szélesedni kezd.

\SubSection{Csomópont átméretezése}

Az átméretezni kívánt csomópontra való dupla kattintással (a szöveg hozzáadásához hasonlóan) érhető el a csomópont átméretezése. Az új méreteket az eszköztár aljában a kattintásra megjelenő felületen lehet megadni.

A téglalap esetén először annak szélességét, majd a magasságát kell megadnunk. A kör alakú csomópontok (circle, start, end) esetében csak a kör sugarát kell megadnunk.
%Ezt át kell nevezni, ha magyarul lesznek leírva a csomópontok nevei

Miután beírtuk az új méreteket, az ezek alatt található \texttt{Resize} gomb megnyomása fogja átméretezni a kiválasztott csomópontot.

\SubSection{Téma megváltoztatása}
% TÉMÁK vagy TÉMA, ezt is el kell dönteni
Az adott témát megváltoztatni az eszköztár alatt található \textbf{Témák} felirat melletti legördülő menüben lehetséges. Egy másik téma választása újabb színekkel látja el a csomópontokat. Természetesen egyszerre több témát is használhatunk. Ilyenkor kiválasztunk egyet, felvisszük a kívánt elemeket, majd témát váltunk, és az új téma új színeivel további elemeket adunk hozzá a folyamatábrához.

\SubSection{Folyamatábra mentése}

Az elkészült folyamatábrát a Témák melletti \textbf{Mentés} gombra kattintva menthetjük el. Ez a funkció minden, a szerkesztő felületen megtalálható elemet (csomópontok, összekötő vonalak, szövegek) elment. Viszont ha a Mentés gomb megnyomása után további szerkesztéseket hajtunk végre, és nem nyomunk ismételten mentés gombot, akkor ez utóbbi módosítások már nem lesznek eltárolva az adatbázisban.

Fontos még megjegyezni, hogy a gráfszerkesztő program a szerver futtatása nélkül is használható, viszont ebben az esetben (értelemszerűen) a mentés funkciót nem tudjuk használni. Ilyenkor (a legtöbb böngészőben) a szerkesztő felületen bárhová kattintva az egér jobb gombjával \texttt{PNG} fájlként menthetjük a folyamatábrát.