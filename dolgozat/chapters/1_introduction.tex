\Chapter{Bevezetés}

A szakdolgozat egy szervezeti folyamatok modellezését segítő gráfszerkesztő alkalmazás elkészítését és működését mutatja be. Egy szervezet életében nagyon sok folyamattal lehet találkozni napi szinten, melyek sokasága és hatékony végrehajtása érdekében célszerű lehet azokat valamilyen folyamatkezelő program segítségével ábrázolni úgy, hogy az ábrákat későbbi használat céljából el is lehessen menteni, majd visszatölteni megtekintés, esetleg módosítás okán.

Több folyamatmodellezésre használható, készen elérhető alkalmazás létezik már. Felvetődhet a kérdés, hogy akkor miért volt szükség még egy elkészítésére a szakdolgozat keretein belül? Többek között erre a kérdésre is választ fogunk kapni a szakdolgozat elolvasása után.

Az első tartalmi fejezet általános áttekintést ad az üzleti (szervezeti) folyamatokról, illetve azok modellezéséről. Ismertetem a témával kapcsolatos fogalmakat, és meg is magyarázom őket. Az ezt követő fejezetek az alkalmazás tervezési folyamatait (alkalmazott technológiáit), megvalósítását (programkódokkal szemléltetve), valamint használatát írja le. Végül, ez utóbbi könnyebb megértése érdekében néhány gyakorlati példa segítségével szemlélteti, hogy hogyan történik a folyamatmodellezés az alkalmazás segítségével.