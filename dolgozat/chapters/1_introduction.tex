\Chapter{Bevezetés}

A szakdolgozat egy szervezeti folyamatok modellezését elősegítő gráf szerkesztő alkalmazást mutat be. Egy szervezet életében nagyon sok folyamattal lehet találkozni napi szinten, melyek sokasága és hatékony végrehajtása érdekében célszerű lehet azokat valamilyen folyamatkezelő program segítségével ábrázolni.

Több folyamatmodellezésre használható, készen elérhető alkalmazás létezik már. Akkor miért volt szükség még egy elkészítésére a szakdolgozat keretein belül? Többek között erre a kérdésre is választ fogunk kapni a szakdolgozat elolvasása után.

Az első tartalmi fejezet általános áttekintést ad az üzleti (szervezeti) folyamatokról, illetve azok modellezéséről. Az ezt követő fejezetek az alkalmazás tervezési folyamatait, megvalósítását, valamint használatát írja le. Végül ez utóbbi könnyebb megértése érdekében néhány gyakorlati példa segítségével szemlélteti, hogy hogyan történik a folyamatmodellezés az alkalmazás segítségével.